
\chapter{Μελλοντικές Επεκτάσεις - Προσωπικά Δεδομένα - Άδειες} % Main chapter title

\label{Μελοντικές Επεκτάσεις - Προσωπικά Δεδομένα - Άδειες} % Change X to a consecutive number; for referencing this chapter elsewhere, use \ref{ChapterX}

\lhead{Κεφάλαιο 6. \emph{Μελλοντικές Επεκτάσεις - Προσωπικά Δεδομένα - Άδειες}} % Change X to a consecutive number; this is for the header on each page - perhaps a shortened title

\section{Μελλοντικές Επεκτάσεις}

\begin{description}
\item \textbf{Βελτιστοποίηση των συστάσεων}  \hfill \\
Στο κεφάλαιο 2 (Σχεδιασμός) όπου περιγράφεται η αρχιτεκτονική του PServer γίνεται λόγος για το \emph{Recommendation Engine}, το ενδιάμεσο επίπεδο μεταξύ αυτού και της εφαρμογής. 
Σε αυτό το στάδιο επεξεργάζονται οι πληροφορίες που δίνει ο PServer ώστε να έχουν νόημα για την εκάστοτε υπηρεσία. 
Είδη έχει ξεκινήσει ο σχεδιασμός ενός τέτοιου μηχανισμού ο οποίος θα μπορεί να εκμεταλλευτεί πλήρως τις δυνατότητες του PServer συνδυάζοντας τις πληροφορίες που δίνονται σε κάθε λογικό
επίπεδό (Personal, Community, Social).
\item \textbf{Χρήση σε άλλες επιστήμες}  \hfill \\
Εμπνευσμένοι από το \emph{Community structure in jazz}\cite{jazz} όπου με \setlanguage{english} collaborative \setlanguage{greek} προσέγγιση από τις πληροφορίες για συνδέσεις των jazz μουσικών και σχημάτων προκύπτουν κοινοτικές
δομές, θα θέλαμε το πρόγραμμά μας να συμβάλει σε κοινωνικές και ανθρωπιστικές επιστήμες. Η χρήση του μπορεί να αναδείξει νέες κοινωνικές ομάδες και ρεύματα, συνεπώς μπορεί να συμβάλει
στην καλύτερη κατανόηση του κοινωνικού μας περιβάλλοντος και κατ επέκταση στην καλύτερη διαχείριση.
\item \textbf{Επικαλυπτόμενες Κοινότητες}  \hfill \\
Μελετώντας την σχετική βιβλιογραφία, η πιο πλούσια και ρεαλιστική προσέγγιση όσον αφορά τις κοινότητες συνδεδεμένων μελών που εγείρει το ενδιαφέρον μας και ανοίγει νέους ορίζοντες για έρευνα 
είναι το άρθρο \emph{Uncovering the overlapping community structure of complex networks in nature and society}\cite{overlapping}.
Οι συγγραφείς του ερμηνεύουν τα δικτυακά συστήματα στη φύση και την κοινωνία ως την συνύπαρξη των μικρότερων κοινοτήτων που τις αποτελούν. 
Αναγνωρίζοντας πως τέτοια σύνολα εμφανίζουν επικαλυπτόμενες κοινοτικές δομές, εισάγουν μετρικές και μεθόδους για την περαιτέρω μελέτη τους, προσπαθώντας να αποκαλύψουν την πραγματική τους φύση
και ξεπερνώντας τις μέχρι τώρα ντετερμινιστικές μεθόδους.
Μακροπρόθεσμος στόχος μας είναι να επεκταθεί η παρούσα εργασία σε ένα σύστημα που θα εκμεταλλεύεται τις πληροφορίες που αντλούνται από ένα περίπλοκο σύστημα με επικαλυπτόμενες
κοινότητες ώστε να εξάγει γνώσει από αυτό.
\end{description} 


\section{Προσωπικά Δεδομένα}
\noindent
Οι τεχνολογίες προσωποποίησης χρησιμοποιούνται πλέον σε πληθώρα διαδικτυακών εφαρμογών
και υπηρεσιών. Αποτελούν ανεκτίμητο εργαλείο αλλά ταυτόχρονα φέρνουν
σημαντικές δεοντολογικές συγκρούσεις στον επιστημονικό κόσμο. 
Αυτό συμβαίνει διότι κατά την δημιουργία των προφίλ χρηστών που προαναφέρθηκαν,
θίγονται ευαίσθητα θέματα και εμφανίζονται ηθικά διλήμματα τα οποία σχετίζονται με την 
\textbf{παραβίαση προσωπικών δεδομένων}.
Η διαδικασία συλλογής των δεδομένων μπορεί να πραγματοποιείται \emph{άμεσα}, 
όπως στις περιπτώσεις που ο ίδιος ο χρήστης καταχωρεί στοιχεία για αυτόν, 
είτε \emph{έμμεσα} όπως στις περιπτώσεις που βγαίνουν συμπεράσματα από την 
αλληλεπίδρασή του με το σύστημα.\cite{eirinaki2003web}\\
Και στις δυο περιπτώσεις ο χρήστης χάνει την ανωνυμία του και επομένως πρέπει να γνωρίζει πως τα προσωπικά
του δεδομένα καταγράφονται και χρησιμοποιούνται από την υπηρεσία.
Ακόμα και στις περιπτώσεις που ο ίδιος έχει επιτρέψει στην υπηρεσία την καταγραφή δεδομένων,
μέσω της χρήσης Cookies, τέτοιου είδους πληροφορίες μπορούν να ανακατευθυνθούν σε 
άλλους διαδικτυακούς τόπους ώστε να υποκλαπούν.

Σε κάθε περίπτωση εφαρμογής τέτοιου είδους τεχνολογιών πρέπει από την αρχή να γίνεται
σαφές στον χρήστη πως οι κινήσεις του καταγράφονται ώστε να φροντίζει μόνος του για την 
επιλογή των δεδομένων που θα γνωστοποιηθούν. Αυτό βέβαια προϋποθέτει πως και τα άτομα που 
διευθύνουν την υπηρεσία είναι άξια εμπιστοσύνης. Ελπίζουμε και επιδιώκουμε τα αποτελέσματα
αυτής της εργασίας να μην χρησιμοποιηθούν με αρνητική σκοπιμότητα άλλα για να 
επηρεάσουν θετικά την ανθρώπινη και την κοινωνική ζωή.

\section{Άδειες}
\noindent Στα πλαίσια της παρούσας πτυχιακής χρησιμοποιήθηκαν τα παρακάτω εργαλεία κατασκευασμένα από 
άλλους προγραμματιστές τα οποία έχουν εκδοθεί κάτω από άδειες ανοιχτού κώδικά.\\



%\begin{center}
\hspace{-9.0em}
    \begin{tabular}{ | p{4cm} | p{6cm} | p{8cm} | }
    \hline
    \textbf{project} & \textbf{Licence} & \textbf{Source} \\ \hline 
    WeakComponent & BSD open-source license & JUNG Software Library \url{http://jung.sourceforge.net/license.txt} \\ \hline  
    EdgeBetweenness & BSD open-source license & JUNG Software Library \url{http://jung.sourceforge.net/license.txt}  \\ \hline
    BronKerbosch & EPL (Eclipse) and LGPL  & jgrapht Java graph Library \url{https://github.com/jgrapht/jgrapht/wiki/Relicensing} \\ \hline
    Metis & Apache License, Version 2.0 & Karypis George \url{https://www.apache.org/licenses/LICENSE-2.0} \\ \hline  
    GPmetis & LGPL Version 3 & Grph - graph library \url{http://www.i3s.unice.fr/~hogie/grph/?page=License} \\ \hline
    betweenness figure \ref{fig:GraphBetweenness} & GNU 1.2, CC 3.0, CC 2.5 & Claudio Rocchini(2007) - Wiki \url{https://en.wikipedia.org/wiki/File:Graph_betweenness.svg} \\ \hline
    PServer & Apache License, Version 2.0 & Dimokritos - SciFy \url{http://pserver-project.org/en/content/pservers-source-code} \\ \hline
    \end{tabular}
%\end{center}