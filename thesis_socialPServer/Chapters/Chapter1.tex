
\chapter{Εισαγωγή} % Main chapter title

\label{Εισαγωγή} % Change X to a consecutive number; for referencing this chapter elsewhere, use \ref{ChapterX}

\lhead{Κεφάλαιο 1. \emph{Εισαγωγή}} % Change X to a consecutive number; this is for the header on each page - perhaps a shortened title
\noindent
Η συνεχόμενη αύξηση του μεγέθους και της χρήσης του διαδικτύου έχει ως επακόλουθο
οι υπηρεσίες να υπερφορτώνονται με πληροφορίες και η χρήση τους να γίνεται
όλο και πιο πολύπλοκη. Οι χρήστες καθημερινά καλούνται να αντιμετωπίσουν
την υπερπληροφόρηση η οποία συχνά μάλιστα τους εμποδίζει από το να εκπληρώσουν τον αρχικό τους σκοπό.
Επομένως γίνεται όλο και πιο επιτακτική η ανάγκη για \emph{προσωποποίηση των υπηρεσιών}.
Προσωποποίηση θεωρείται η δυνατότητα μιας υπηρεσίας να λαμβάνει υπόψιν τις προτιμήσεις του κάθε χρήστη 
κατά την διαμόρφωση του περιεχομένου της.

\section{Ανάλυση Του Προβλήματος}
\noindent
Η παρούσα εργασία ξεκινά με μια βασική υπόθεση: Έχουμε στη διάθεσή μας μια \emph{γενικευμένη μηχανή προσωποποίησης περιεχομένου} 
\setlanguage{english} \emph{(\hyperref[PServer]{PServer 3.5})} \setlanguage{greek}.
Πρόκειται για έναν \emph{εξυπηρετητή} ο οποίος όπως και κάθε μηχανή που προσφέρει
προσωποποίηση, λαμβάνει υπόψιν τις προτιμήσεις του κάθε χρήστη και 
ακολούθως διαμορφώνει το περιεχόμενό της σύμφωνα με τα 
χαρακτηριστικά του κάθε ενός.
Στην περίπτωση ενός δικτυακού τόπου ο PServer είναι ένας ακόμα \emph{εξυπηρετητής} 
ο οποίος αναλαμβάνει να προσωποποιήσει οποιαδήποτε υπάρχουσα υπηρεσία,
συλλέγοντας πληροφορίες για τους χρήστες και την αλληλεπίδραση τους με αυτήν και τελικά παρέχοντας τις πληροφορίες που χρειάζονται για την προσωποποίηση. 
Η διαδικασία αυτή για να πετύχει τους στόχους της εμπεριέχει ένα πλήρες \emph{Σύστημα Συστάσεων}. 
Για την λειτουργία του, διατηρεί προφίλ χρηστών τα οποία τροφοδοτεί με δημογραφικές πληροφορίες και πληροφορίες προτιμήσεων και στη συνέχεια μπορεί να προβλέψει: 
από όλο το εύρος των πιθανών προτιμήσεων ποιο είναι το υποσύνολο στο οποίο έγκειται ο κάθε χρήστης. 
Τελικώς, μπορεί να κάνει συστάσεις αντικειμένων στους χρήστες βασιζόμενος στην υπάρχουσα αλληλεπίδρασή τους με το σύστημα.

Καλούμαστε να επεκτείνουμε αυτό το σύστημα ώστε να δημιουργεί συστάσεις βασιζόμενο, όχι μόνο στις πληροφορίες των δημογραφικών στοιχείων και των προτιμήσεων, 
αλλά να μπορεί να προβλέπει την συμπεριφορά του κάθε χρήστη μελετώντας την κοινωνική δικτύωση των χρηστών της εφαρμογής. 
Με τον τρόπο αυτό αναμένουμε από το σύστημα να πριμοδοτεί περιεχόμενα με τα οποία είχαν
επαφή οι φίλοι κάποιου χρήστη έναντι περιεχομένων που έχουν δει μόνο άγνωστοι σε αυτόν χρήστες.
Για την εφαρμογή των παραπάνω, σχεδιάστηκε και υλοποιήθηκε ένα σύστημα (\textbf{socialPServer}) το οποίο επιτρέπει τον εντοπισμό των ενδιαφερόντων ενός 
χρήστη εξάγοντας συμπεράσματα από τα αντικείμενα 
τα οποία προτιμούν οι κοντινότεροι φίλοι του,
δοσμένου του κοινωνικού γράφου της εφαρμογής. 
Το σύστημα αυτό μπορεί να χαρακτηριστεί ως \textbf{Social Recommendation System}.

\section{Προσέγγιση}
\noindent
Τα \emph{κοινωνικά δίκτυα} από τη φύση τους αποτελούν πολύ σημαντική πηγή εξόρυξης δεδομένων, καθώς η δομή τους εμπεριέχει πληθώρα πληροφοριών για τα μελή τους.
Όταν μια εφαρμογή διατηρεί τον κοινωνικό γράφο των χρηστών της, μπορεί να εξάγει πληροφορίες οι οποίες ξεφεύγουν 
από τα πλαίσια της υπηρεσίας και αποκαλύπτουν στοιχεία για την πραγματική ζωή και συμπεριφορά των χρηστών. 
Αυτό προκύπτει από το γεγονός ότι ο άνθρωπος, ο οποίος από την φύση του είναι κοινωνικό όν, σε όλη την διάρκεια 
της ζωής του τείνει να ετεροκαθορίζεται και κατ' επέκταση να δρα επηρεασμένος από το κοινωνικό του περιβάλλον.
Έτσι ο τομέας του \emph{Social Analysis} αποκτά πολύ μεγάλη σημασία για την έρευνα αλλά και τις σημερινές ηλεκτρονικές υπηρεσίες.
Πλέον η εξαγωγή γνώσης δεν εξαρτάται μόνο από την αλληλεπίδραση του χρήστη με το σύστημα, αφού ο κάθε χρήστης δεν αντιμετωπίζεται σαν μονάδα αποκομμένος από το σύνολο 
αλλά σαν μέλος του. 
Η επίδραση που έχει η κοινότητα στο κάθε μέλος βεβαίως είναι αμφίδρομη διαδικασία αφού όπως ο κάθε χρήστης επηρεάζεται από το γύρο περιβάλλον του, έτσι και ολόκληρο το
κοινωνικό σύνολο διαμορφώνεται τελικά από τον κάθε ένα αυτόνομο χρήστη.

Για την πραγματοποίηση των παραπάνω, οι χρήστες της εφαρμογής ομαδοποιούνται σε κοινότητες χρηστών των οποίων τα μέλη έχουν μεταξύ τους ισχυρούς δεσμούς φιλίας. 
Έτσι σε επόμενο χρόνο μπορούν να σκιαγραφηθούν οι προτιμήσεις του κάθε χρήστη μέσω της ανάλυσης των συμπεριφορών και των προτιμήσεων των μελών της κοινότητας στην οποία ανήκει. 
Μέσω αυτής της διαδικασίας ο pServer θα είναι πλέον σε θέση να κάνει συστάσεις σε κάποιο χρήστη βασιζόμενος στις προτιμήσεις των κοντινότερων φίλων του
από τους οποίους, όπως και στην πραγματική ζωή, εκτιμάται ότι δέχεται και την υψηλότερη επιρροή.

